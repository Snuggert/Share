\documentclass[pdftex,12pt,a4paper]{article}
\usepackage{wrapfig,amssymb,amsmath,graphicx,color, listings, verbatim}
\pagenumbering{arabic}
\newcommand{\HRule}{\rule{\linewidth}{0.5mm}}

\begin{document}
\begin{titlepage}

% Upper part of the page
\begin{flushleft}
\includegraphics[trim=23mm 0mm 0mm 0mm, width=1.2\textwidth]{./logo}\\[1cm]   
\end{flushleft}
\begin{center}
\textsc{\Large Data Structures}\\[0.5cm]


% Title
\HRule \\[0.4cm]
{ \huge \bfseries Maze Solver }\\breadth first

\HRule \\[1.5cm]

% Author and supervisor
\begin{minipage}{0.4\textwidth}
\begin{flushleft} \large
\emph{Author:}\\
 Abe \textsc{Wiersma}
\end{flushleft}
\end{minipage}
\begin{minipage}{0.4\textwidth}
\begin{flushright} \large
\end{flushright}
\end{minipage}

\vfill

% Bottom of the page
{\large \today}

\end{center}
\end{titlepage}
\topmargin=-0.5in 
\textheight=25cm
\pagebreak

\tableofcontents
\pagebreak

\section{Preface}
For the course data structures we got the assignment to make a maze solver which finds the shortest path, this is my execution of the assignment.
For this assignment we got some skeleton code for reading a maze, and code to create a spanning tree. 

\section{Problems}
To make the maze solver there are a few problems to be solved, this section shall contain all the problems that needed solving.
\subsection{Reading the maze file}
For this assignment we got maze files in this format:
\lstinputlisting{map3.txt}
So reading files like this is the first problem.
\subsubsection{Finding the shortest path}
Well this one speaks for itself really, but there are several different ways to do this.

\section{Solutions}
And now my solutions .
\subsection{Reading the maze file}
Because this time the skeleton code provided had a fully operational maze reader which reads a maze into a struct:\\
- The maze itself is stored in a two dimensional char array called map.\\
- The dimensions of the array are stored in the int's nrows and ncols.\\
Because I wanted the struct to store the position of the exit and the start i added a function in the maze.c and I added four int's which store de positions of the start and exit. 

\subsubsection{Finding the shortest path}
The algorithm I used to find the shortest path is the breadth first algorithm. I made this algorithm for assignment 3 which also involved solving a maze. I shall be explaining it with visual help.
\begin{verbatim}
########
#E     #
###S####
##   ###
########
\end{verbatim}
This will be our explanation maze.

The algorithm starts of by making a 2-d int-array with the same size of the char array in the struct. All values in this 2-d array are set to -1.
\begin{verbatim}
-1-1-1-1-1-1-1-1	########
-1-1-1-1-1-1-1-1	#E     #
-1-1-1-1-1-1-1-1	###S####
-1-1-1-1-1-1-1-1	##   ###
-1-1-1-1-1-1-1-1	########
\end{verbatim}
Then the start position gets defined in the int Array, this is step 0.

\begin{verbatim}
-1-1-1-1-1-1-1-1    ########
-1-1-1-1-1-1-1-1    #E     #
-1-1-1 0-1-1-1-1    ###0####
-1-1-1-1-1-1-1-1    ##   ###
-1-1-1-1-1-1-1-1    ########
\end{verbatim}
Now the looping begins until the end is found. The int array changes like these steps...

\begin{verbatim}
-1-1-1-1-1-1-1-1    ########
-1-1-1 1-1-1-1-1    #E 1   #
-1-1-1 0-1-1-1-1    ###0####
-1-1-1 1-1-1-1-1    ## 1 ###
-1-1-1-1-1-1-1-1    ########
\end{verbatim}

\begin{verbatim}
-1-1-1-1-1-1-1-1    ########
-1-1 2 1 2-1-1-1    #E212  #
-1-1-1 0-1-1-1-1    ###0####
-1-1 2 1 2-1-1-1    ##212###
-1-1-1-1-1-1-1-1    ########
\end{verbatim}

\begin{verbatim}
-1-1-1-1-1-1-1-1    ########
-1 3 2 1 2 3-1-1    #32123 #
-1-1-1 0-1-1-1-1    ###0####
-1 3 2 1 2-1-1-1    ##212###
-1-1-1-1-1-1-1-1    ########
\end{verbatim}
At this point the end is found and the looping stops.
You can see the int array respecting the walls in the char array.

So at this point there are numbers 0 to 3 making a path to the exit. Because the location of the exit is stored we can easily walk back number for number from this position.
This is done like so.

\begin{verbatim}
########
#E.    #
###S####
##   ###
########
\end{verbatim}

\begin{verbatim}
########
#E..   #
###S####
##   ###
########
\end{verbatim}

In a large maze this would look like this.... (this example is the solution of map3.txt given to us with the assignment.)
\begin{verbatim}
######################################
#E# #        #       ## #     # ##   #
#...# # ####   #####    ## ## # #  # #
###.###  # #####   ####  #  #     ## #
#S#...##...      #    ## ## ######## #
#.###..#.#.####### ## ## #  #    ##  #
#.# ##...#.##    ###   #   ## ##  ## #
#..  ### #... ##   #########   ## #  #
##.### #####.##  #           #  ### ##
# ......#.... #### ## # #######  #   #
#######...# #   ####  # #    ### # ###
#     ##### ###    #### # ##  #  #   #
# ###    #  # ####   #  #  ## # #### #
# # ####   ##   ####   ### #  #    # #
# #   ######  #    #####   # ## #### #
# # #     ### #### #     # ###   ##  #
# # #####   ###    ### ###  #  # #  ##
# #  # ####  ### #   # # ## # #### ###
# ## #  # ## #   ### # #  # #   ## # #
#  # ##    #   #  ##   ## # ###  # # #
## # ## # #### ##  ######   # ## #   #
## # #  #   #  ###  #   ## ##  # ### #
#  # #### # # ## #### #  #    ## # # #
# ##    # ###     #   ## #### ##   # #
#  # ## # #   # ### ####    ###  ### #
## # #  # # #####   #  #### #   ##   #
#  # # ## #  #    ### ##  #   # #  ###
# #### #  ## # #### #    ###### # ## #
#    # # ##  # #    ## # #   #### #  #
# ## # #    ##   ##    # # #    # ## #
# #  # ############# # ### ### ##    #
### ##      #     ####  #   #   #### #
#   ## ## #   # #  ##  ## ### #    # #
# ###   ##########  # ##  # # #### # #
#   #####   ##   ##   ## ## #   #### #
# #    ## #  # # ######     ###    # #
# # ##    ##   #        # #   # ##   #
######################################
\end{verbatim}
\section{Conclusion}
In the skeleton code we were provided with a way to implement a tree which I started of with, but after a while I thought the simplicity of my previous algorithm had its charm, and I recoded it to work with the new maze struct.
I was pretty far advanced with the implementing of the tree:
I had thought of a way to implement an array of nodes similair to the int array, but not to fill the walls with a value as I did with the int array. Then to follow the path back from the exit node to the start node parent after parent.

In the archive i will also put two generated mazes given to me by my good friend Pjotr Buys whom made a generator, this to show it also works with mazes of a size 1000x1000, and just a proof of concept.


\end{document}